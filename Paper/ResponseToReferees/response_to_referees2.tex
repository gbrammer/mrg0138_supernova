\documentclass[11pt]{article}
\usepackage[usenames,dvipsnames]{xcolor}
\usepackage{soul}
\usepackage{graphicx}
\usepackage{lineno}
%\linenumbers

\usepackage{hyperref}
\usepackage{booktabs}

%\usepackage{amssymb}
\usepackage{xspace}
\usepackage{graphicx}
%\usepackage{todonotes}
\usepackage{ifthen}

\usepackage{geometry}

\topmargin 0.0cm
\oddsidemargin 0.2cm
\textwidth 16cm 
\textheight 21cm
\footskip 1.0cm

% Remove the "Contents" label from the Table of Contents
\makeatletter
\renewcommand\tableofcontents{%
    \@starttoc{toc}%
}
\makeatother

\begin{document}

\newif\ifone
\newif\iftwo
\newif\ifthree
\newif\ifed
\newif\ifhlt

\onetrue % Response to referee 1
%\onefalse
\twotrue % response to referee 2
%\twofalse
\threetrue % response to referee 3
%\threefalse
\hlttrue  % Highlighting page number checks
%\hltfalse

\ifone
\newcommand{\one}[2]{\textbf{\textcolor{blue}{ #1}}\\ \linebreak {#2}\\\noindent\rule{\textwidth}{1pt}\bigskip}
\else
\newcommand{\one}[2]{}
\fi

\iftwo
\newcommand{\two}[2]{\textbf{\textcolor{teal}{ #1}}\\ \linebreak {#2}\\\noindent\rule{\textwidth}{1pt}\bigskip}
\else
\newcommand{\two}[2]{}
\fi

\ifthree
\newcommand{\three}[2]{\textbf{\textcolor{Maroon}{ #1}}\\ \linebreak {#2}\\\noindent\rule{\textwidth}{1pt}\bigskip}
\else
\newcommand{\three}[2]{}
\fi

\ifed
\newcommand{\ed}[2]{\textbf{\textcolor{DarkOrchid}{Ed: #1}}\\ \linebreak {#2}\\\noindent\rule{\textwidth}{1pt}\bigskip}
\else
\newcommand{\ed}[2]{}
\fi

\ifhlt
\newcommand{\hlt}[1]{\hl{#1}}
\else
\newcommand{\hlt}[1]{{#1}}
\fi



\begin{center}
\textbf{\Large{Response to Referees}}\\
\smallskip
\today
\end{center} 
\medskip


Our point-by-point responses are presented below. In this document we have grouped the responses by reviewer.   We have color-coded each referee's comments, matching the colors used in the changetext version of the manuscript, as follows:

\begin{itemize}
\item \textbf{\textcolor{blue}{Items from Referee \#1 in blue.}}
\item \textbf{\textcolor{teal}{Items from Referee \#2 in teal.}}
\item \textbf{\textcolor{Maroon}{Items from Referee \#3 in maroon.}}
\end{itemize}


\pagebreak


\ed{As you will note, there is still a range of different opinions regarding your paper. Referee \#1 has some remaining minor concerns, however it their comments to the editor express concern about the dependence of your results on the lens modelling and how that dependence might compromise the confidence and universality of your results.\\ \smallskip\\
More seriously, referee \#2 is still questioning the relevance of your results in the context of precisions cosmology. Particularly, while they find your analysis interesting, they disagree that the proposed method is a valid tool for precision cosmology for the next 10 years.\\ \smallskip\\
Combined with the comments of referee \#1, I think it will be important to try and more accurately reflect the applicability, relevance and robustness of your method. One way to proceed would be to follow the suggestions of referee \#2 and significantly tone down your claims regarding precision cosmology and instead focus on the uniqueness of this system more generally.\\ \smallskip\\
Given the current set of reports, it will be important to convince referee \#2 if we are to consider your manuscript for publication.
}{}


\clearpage


\section{Reviewer 1 (Comments for the Author):}

\one{The authors have incorporated most of the previous comments. I have a few more comments (in no particular order) after which the manuscript could be considered for publication.\\ 
\linebreak
1. The authors have made use of positional data constraints alone, have not used data constraints from images 4-5 and have assumed one specific lens density profile in this analysis. All of these are known to affect the main predictions which are the magnifications and time delays. This is still a point of concern. The current uncertainties do not account for that. Therefore, the readers need to be sufficiently warned about the biases and uncertainties, as a result of the above. This could be done by quantifying the uncertainties to the extent possible or by adding cautionary statements in the main text and also in the abstract, at minimum.
}{
We do not disagree with the sentiment of the referee on this point.  Systematic lens modeling uncertainties are absolutely an essential component for understanding how seriously to take the conclusions about the SN classification and time delay estimation presented here.
In this work we have focused on presenting the analysis using only truly {\it blind} lens models, developed without knowing the impact of lens modeling choices on the SN classification or time delay. As a result, we acknowledge that we have only a limited window on potential systematic biases.

A rigorous quantification of MACS J0138 lens modeling uncertainties would be valuable, but it is a significant undertaking.  Ideally it would involve multiple lens modeling teams using different software tools and modeling methodologies (e.g., as has been done with MACS J1149 for SN Refsdal, or with the Frontier Fields simulated cluster lens modeling challenge of Meneghetti et al. 2017).   We feel this is beyond the scope of this work, so we've attempted to address the referee's concern with appropriate caveats throughout.  These are necessarily brief in the abstract \hlt{(blue text, lines 18-20)} and main text \hlt{(lines 84-86)}, but are expanded upon in the methods \hlt{(lines 343-346)}, and supplemental materials \hlt{(lines 514-521)}.
}


\one{2. Fig. 5 shows model-predicted images of the host galaxy for images 1-3 but not for images 4 and 5 which can be quite important in deciding the robustness of the model E.}{
  (note that this is now Fig 4)\\  
\textcolor{red}{TODO: Johan will add this to the figure}\\

We have added this to Figure 5.  \textcolor{red}{TODO: add some comments here and in the supplemental text about how well images 4 and 5 are reproduced.  If appropriate, we could note that Newman+ 2018 said ``We note that the radial images 4 and 5 were not used to constrain the model except via their approximate positions, so a close match to their detailed structure is not expected.''
}}

\one{3. Some of the best model parameters reported in Table 5 still look odd. For example, a velocity dispersion of 700 km/s for the BCG seems unrealistic.}{  
\textbf{\textcolor{red}{Johan: please review and edit as needed}}\\

This is certainly a fair point.  In the main text the new sentence at \hlt{lines 87-89} now indicates that we did not include stellar kinematics constraints. In the first resubmission we added a paragraph to the ``Future Lens Modeling'' section \hlt{(p.39)} about the BCG velocity dispersion measurement and its implications for lens modeling. We have added a new sentence in that section \hlt{(lines 540-541)} to highlight that our adopted value for the BCG velocity dispersion is at odds with our measurement from the VLT data ($390\pm10$ km/s).   
As described there, we have chosen to limit the main conclusions of this paper to only the blind models, and therefore did not develop a new lens model that incorporates the measured BCG velocity dispersion.  We look forward to exploring that in future work.
}

\one{4. Model predicted magnification ratios and measured magnification ratios (image2 to image1) are not consistent.}{
We are unsure of exactly what the referee is referring to here.   If this refers to the magnification ratios of the SN images, then there is a misunderstanding, because we have not attempted to measure any SN magnification ratios.  Given the limited data,  only the color-curve fitting method can be used to measure SN lensing properties independently of the lens model.  The SN color curves are insensitive to "macro-lensing" magnification, so we can't measure $\mu$ ratios from them.  For all other SN fitting, we adopt the magnifications from the lens model, as described in the Methods.\\

This comment may instead refer to the host galaxy image magnifications, as reported in Supplementary Table 1.  This may again be a misunderstanding.  The values in that table are all predicted magnifications, either from our lens model (columns 2-4) or from the Newman+ 2018 model (column 5).  As described in the Supplementary Note text, the relevant comparison here is from column 4 ($\mu_{gal,avg}$) against column 5 ($\mu_{N18,gal,avg}$).   The $\mu$ ratios (and the individual $\mu$ values) are fully consistent between those two models.
}

\one{5. Table 1 and Table 6 have duplicate information. Authors could merge them.}{
Thank you for the suggestion.  We've merged them. \\
\textcolor{red}{Johan:  can you give the predicted RA, Dec coordinates for image SN5?}
}

\one{6. The authors mention that systematic uncertainties are accounted for in the magnification uncertainties. However, these do not include biases due to the assumption of the density profile which can strongly affect magnifications. This should be clearly mentioned.}{
This is now addressed directly in the main text \hlt{(lines 84-86)}, the methods \hlt{(lines 343-346)}, and supplemental materials \hlt{(lines 514-521)}.

\textcolor{red}{Johan?  can you recommend a good reference to cite for this?
   Maybe Birrer et al 2020?  (that's more about the mass-sheet transform though)}
}

\clearpage
\section{Reviewer 2 (Comments for the Author):}

\two{While authors have improved the manuscript in many ways, and I am still in favor of having the discovery reported in Nature Astronomy, my main concern with respect to precision cosmology persists. As indicated in their Fig 3, the limiting factor in time-delay cosmography is the lens model (in their case 2 or 5\% uncertainty assumed). Thus, the current emphasis on the power of waiting for 20 years to maybe catch a SN image that would allow for, potentially, a sub percent time delay measurement seems rather misplaced.\\
The system of AT 2016tbd, as pretty as it is, it does not have any more power in measuring the Hubble constant than any other system with just $\sim$100 days time-delay and 1-2 days accuracy. As long as Fig 3 remains in the manuscript, I cannot recommend publication. It just is not relevant and gives the wrong impression.}{


We will concede this point.  It is certainly true that the 20-year timeline makes any cosmological forecasting highly speculative.  It is also of course true that any lensed SN with a $\sim$100-day time-delay measured with a precision of 1-2 days will be equivalent to a long-delay system like this one. It may be that we diverge from the referee primarily on the question of how many lensed SN like that could be observed in the coming decade, and at what cost.  Evaluating that (and doing a more rigorous cosmological analysis) is beyond the scope of this discovery paper---and we appreciate that Reviewer \#2 does agree with us on that.

We have removed Figure 3 (the projected cosmological contours) and the related discussion from the main text.  In its place, we've added a paragraph that introduces a little of the ``cost/benefit analysis'' discussion, comparing the observational cost of SN Refsdal against AT 2016tbd \hlt{(lines 151-157)}.  We also added a few sentences discussing the cosmological utility of cluster-lensed SN systems more generally, which also addresses a request from Referee \#3 \hlt{(red text, lines 163-167)}.
}

\subsection{Some minor points:}

\two{Abstract:
``Achieving these cosmological goals requires many more lensed supernova discoveries and more efficient observational follow-up.''
\linebreak
I think the authors are getting ahead of themselves, “more” here is quite premature, as they have not described what there is yet.}{
Fair enough.  We've removed ``more'' from the abstract.}

\two{``This object demonstrates that cosmologically useful time delays can be measured with minimal observational cost using cluster-lensed supernovae.''
\linebreak
I do not quite get the “minimal” part here. It would be a major effort for a space mission to hunt for the 4:th image over 4 years!}{For simplicity, we've removed this statement, but we'll also take a moment to explain: This statement was intended to refer to the fact that when the time delay is long enough, the SN measurements can be anchored at either end by just a few observations, making the whole time-delay measurement less expensive.  
We've added some text in the supplemental note on Future Observations that we hope will clarify our intended meaning \hlt{(lines 578-587)}.  In the previous resubmission we did also include a discussion of the observational cost for a monitoring campaign to catch the reappearance \hlt{(lines 557-566).}
}


\two{Introduction:
While describing the H0 tension, the authors should certainly clarify that the community is also engaged in exploring potential systematic effects behind the tension.}{
 Certainly.  We have added a few sentences \hlt{(teal text, lines 32-34)}, but have kept the Verde+ 2019 reference. It is a conference proceeding, and a little old in this fast-moving field, but its one of the few that presents the search for systematic biases and the hunt for new physics with some balance.
}

\two{I also wonder why Ref [12] is singled out here. It certainly was not the first, see e.g., \href{https://ui.adsabs.harvard.edu/abs/2002A\%26A...393...25G/abstract}{2002A\&A,393,25}
(and there may be others, I have not made an exhaustive search).}{
This is a very fair point.  We had been trying to finesse the 30-reference Nature Astronomy limit by identifying appropriate references that can do ``double duty.''  In an earlier draft we were applying Ref [12] (Coe \& Moustakas 2009) both in the introduction and in the description of future cosmological constrains (the former Figure 3).  

To replace that reference, we have settled on Holz 2001 as the first to explicitly connect time delay cosmography with measurement of {\em dark energy parameters}.  One could certainly argue for inclusion of others, but we are right against the limit. 
}



\clearpage

\section{Reviewer 3 (Comments for the Author):}

\three{I thank the authors for addressing my initial comments, along with those from the other referees. I am satisfied with the changes they have made to the paper, and recommend it for publication. I do have a few very minor comments that the authors may wish to take under advisement before submitting the final version.\\
---\\
Page 3 (2nd paragraph): The time-delay cosmography method has now been applied to seven lensed quasars (see Millon et al. 2020, Birrer et al. 2020).}{
Corrected.
}

\three{Page 7: In their response to my initial comments, the authors acknowledge that lens model systematics may limit the uncertainties, but argue that this method is valuable regardless. I agree with this assessment, but the paper text doesn't really acknowledge this, and presents the 5\%/2\% values as if those are expected uncertainties from current/future samples. I think there should be some additional text here so as not to present an overly optimistic picture of future constraints.}{
This section has been removed, as Reviewer \#2 strongly objected to the cosmological forecasting section.  With the 5\%/2\% values no longer referenced, we have instead included an abbreviated version of our prior response to Reviewer \#3 on this point, in the new maroon text \hlt{(lines 166-170).}
}

\three{Figure 1: The tick marks seem to have disappeared from this figure.}{
Slippery little ticks. We have corrected the figure.

\textcolor{blue}{Gabe: can you redo this fig with ticks?}
}

\three{Figure 3: The left and right columns are labeled "FlatwCDM" and "Flatw0waCDM", respectively, but aren't the same cosmologies being presented in both cases? It's not entirely clear that you need to label the top row, as these have fixed w0 and wa, and thus are essentially flat LCDM. Shouldn't the top row be labeled "Flat LCDM" and the bottom panels be labeled "Flatw0waCDM"?}{
This figure was removed, responding to the strong concerns of Reviewer \#2.  

Though the point is moot, this was due to an error on our part (a misplacement of the w0,wa labeling text)  and our mistake in not explaining clearly enough how the simulations fixed the ``true'' dark energy equation of state values while nevertheless allowing them as free parameters. 
}

\three{Figure 5: The contours in the first column are a bit difficult to see, particularly the orange ones. I would recommend that the authors try to find some colors/linestyles to make these more apparent.}{
\textcolor{blue}{Johan: can you fix, please?}
}

\three{Time Delay Estimation (page 27), 2nd paragraph: I'm not familiar with SALT2 or how these light curve fittings are performed, but I take it that the "stretch" parameter x1 is some parameter of the model light curve that affects its shape, which the authors set to x1 = 0. They state that fixing it to other values from $-1 < x_1 < 1$ results in a change in the time delay by less than five days. Is this explicitly included in the time delay uncertainties provided? Granted, this would not change the results substantially, but it would be good to know for completeness.}{
The referee has a correct understanding of our procedure.   We did not include the variation of a fixed x1 value in the reported time delay uncertainties.  We have now explicitly stated this in the Methods \hlt{(lines 416-418).}  It is effectively a (small) systematic uncertainty now, and we anticipate that it will be substantially reduced with any future analysis. As we noted in the previous response: \\
``we expect that a full light curve of the fourth SN image will provide a tight constraint on x1, which can then be propagated back into the fitting of the SN images 1-3.''\\
}


\end{document}