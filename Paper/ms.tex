\documentclass[twocolumn]{aastex63}
\usepackage{apjfonts}

\usepackage{lipsum}  

%\received{\today}
%\revised{\today}
%\accepted{\today}
%% Command to document which AAS Journal the manuscript was submitted to.
%% Adds "Submitted to " the argument.
%\submitjournal{AJ}

%\topmargin 0.6in

\usepackage{multirow}

%\usepackage{deluxetable}
%\usepackage{color}
%\usepackage[usenames, dvipsnames]{color}
%\definecolor{citeRGB}{rgb}{0,0.1,0.7}
%\usepackage[hyperfootnotes=true,naturalnames=true,letterpaper,pdfstartview=FitH,pdfpagemode=UseNone,colorlinks=true,citecolor=citeRGB]{hyperref}

\gdef\HST{\textit{HST}}
\gdef\Spitzer{\textit{Spitzer}}
\gdef\fluxcgs{\mathrm{erg~s^{-1}~cm^{-2}}}
\gdef\micront{$\mu$m}
\gdef\micronm{\mu\mathrm{m}}
\gdef\aXe{\texttt{aXe}}
\gdef\flux_radius{\textsc{flux\_radius}}
\gdef\epers{\textit{e}$^{-}$ s$^{-1}$}

\gdef\logOH{12 + \log \left(\mathrm{O/H}\right)}
\gdef\SFRuvir{SFR_\mathrm{UV+IR}}
\gdef\peryr{\mathrm{yr}^{-1}}
\gdef\kms{km\,s$^{-1}$}
\gdef\mum{$\mu\mathrm{m}$}
\gdef\24mum{$24\,\mu\mathrm{m}$}
\gdef\arcsec{^{\prime\prime}}
\gdef\UDFj{UDFj-39546284}
\gdef\compareID{UDF-40106456}
\gdef\Lya{\mathrm{Ly}\alpha}
\gdef\Halpha{\mathrm{H}\alpha}
\gdef\Hbeta{\mathrm{H}\beta}

\newcommand\xxx{{\textcolor{red}{\bf xxx}}}
\newcommand\xref[1]{{\textcolor{red}{\bf (REF #1)}}}
\newcommand\XXX[1]{{\textcolor{red}{xxx #1}}}

\gdef\HAWKI{\mbox{HAWK-I}}

\gdef\sncode{\mbox{ABC}}

\shortauthors{Brammer et al.}
\shorttitle{CHArGE}
% \setwatermarkfontsize{50pt} 
%\citestyle{aa}
\begin{document}

\title{SN-\sncode: A Gravitationally-Lensed Type Ia Supernova Candidate at $z=1.9$}

% \footnotetext[*]{Based on observations made with the NASA/ESA \textit{Hubble
% Space Telescope}, programs GO-11640, 12177 and 12328, obtained at the
% Space Telescope Science Institute, which is operated by the Association of
% Universities for Research in Astronomy, Inc., under NASA contract NAS
% 5-26555.}

\correspondingauthor{Gabriel Brammer}
\email{gabriel.brammer@nbi.ku.dk}

\author[0000-0003-2680-005X]{Gabriel Brammer}
\affiliation{The Cosmic Dawn Center, University of Copenhagen, Vibenshuset, Lyngbyvej 2, DK-2100 Copenhagen, Denmark}

%%%%%%%%%%%%%%%%%%%%%%%%%%%%%%%%%%%%%%%
%%%%%%  Abstract
%%%%%%%%%%%%%%%%%%%%%%%%%%%%%%%%%%%%%%%
\begin{abstract}

We report the discovery of a gravitationally-lensed, multiply-imaged transient source with a likely host at $z=1.905$, observed in archival \textit{Hubble Space Telescope} near-infrared imaging observations.  

\end{abstract}
\keywords{supernovae: individual SN-\sncode --- gravitational lensing: strong --- galaxies: evolution --- galaxies: high-redshift}

%%%%%%%%%%%%%%%%%%%%%%%%%%%%%%%%%%%%%%%
%%%%%% Introduction
%%%%%%%%%%%%%%%%%%%%%%%%%%%%%%%%%%%%%%%
\section{Introduction}
\label{s:introduction}

%%%%%%%%%%%%%%%%%%%%%%%%%%%%%%%%%%%%%%%
%%%%%% Archival data summary
%%%%%%%%%%%%%%%%%%%%%%%%%%%%%%%%%%%%%%%
\section{Observations}
\label{s:observations}

%%%%%%%%%%%%%%%%%%%%%%%%%%%%%%%%%%%%%%%
%%%%%%  Lens model
%%%%%%%%%%%%%%%%%%%%%%%%%%%%%%%%%%%%%%%
\section{Strong Lensing Analysis}
\label{s:lensing}

\subsection{Multiple images}
\label{ss:images}

\subsection{Time delay}
\label{ss:timedelay}

%%%%%%%%%%%%%%%%%%%%%%%%%%%%%%%%%%%%%%%
%%%%%%  SN Classification
%%%%%%%%%%%%%%%%%%%%%%%%%%%%%%%%%%%%%%%
\section{Transient Classification}
\label{s:classifiation}

\subsection{Light curve analysis}
\label{ss:lightcurve}

\subsection{Host galaxy properties}
\label{ss:host}

Discussion from \cite{Newman2018a} and \cite{Newman2018b}.

%%%%%%%%%%%%%%%%%%%%%%%%%%%%%%%%%%%%%%%
%%%%%%  Discussion & Conclusions
%%%%%%%%%%%%%%%%%%%%%%%%%%%%%%%%%%%%%%%
\subsection{Discussion}
\label{s:discussion}

%%%%%%%%%%%%%%%%%%%%%%%%%%%%%%%%%%%%%%%
%%%%%%  End things
%%%%%%%%%%%%%%%%%%%%%%%%%%%%%%%%%%%%%%%
\acknowledgments

\bibliography{ms}{}
\bibliographystyle{aasjournal}

Software acknowledgements: astropy, scipy, matplotlib, sep, astroquery, shapely

The Cosmic Dawn Center (DAWN) is funded by the Danish National Research Foundation under grant No. 140. 

\end{document}
